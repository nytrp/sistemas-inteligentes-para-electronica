\documentclass[conference]{IEEEtran}
\usepackage[utf8]{inputenc}
\usepackage[T1]{fontenc}
\usepackage[spanish]{babel}
\usepackage{amsmath,amssymb}
\usepackage{graphicx}
\usepackage{array}
\usepackage{booktabs}
\usepackage{hyperref}

\title{Clasificador de Colores con TCS34725, IQR, EEPROM y KNN en Arduino (Optimizado)}
\author{
  \IEEEauthorblockN{Alejandro Martinez\\Daniel Santiago\\Yurliana Niebles\\Isaac Montes}
  \\
  \IEEEauthorblockA{\IEEEauthorrefmark{1} Universidad De la Costa \\
  \texttt{}}
}

\begin{document}
\maketitle
\begin{abstract}
Se presenta el diseño, implementación y análisis de un sistema embebido basado en Arduino que utiliza el sensor de color TCS34725, filtrado robusto mediante IQR (Interquartile Range), almacenamiento en EEPROM y un clasificador KNN manual optimizado. El objetivo del sistema es reconocer tres clases de color (ROJO, NARANJA, VERDE) con bajos requerimientos de memoria y procesamiento. Se describen las decisiones de diseño, las fórmulas matemáticas usadas y una propuesta de flujo experimental.
\end{abstract}

\begin{IEEEkeywords}
TCS34725, IQR, EEPROM, KNN, Arduino, centroides, clasificación de color
\end{IEEEkeywords}

\section{Introducción}
La clasificación de colores con sensores embebidos es una tarea común en robótica y sistemas de visión industrial de bajo coste. Este trabajo describe una implementación práctica que busca conservar la memoria y CPU del microcontrolador al tiempo que mantiene robustez ante lecturas ruidosas. El sensor TCS34725 provee medidas crudas R/G/B y un canal de claridad C que se emplea para normalizar.

\section{Objetivo}
Implementar y documentar un sistema que:
\begin{itemize}
  \item Tome múltiples lecturas por muestra y elimine outliers mediante IQR.
  \item Acumule muestras por clase y calcule centroides (media) de forma eficiente en memoria.
  \item Guarde centroides en EEPROM para uso posterior por el clasificador KNN.
  \item Clasifique lecturas rápidas (1 muestra) e informe probabilidades por clase.
  \item Mida el tiempo total (lectura + clasificación) por petición.
\end{itemize}

\section{Marco teórico}
\subsection{Sensor TCS34725}
Sensor de color que entrega valores crudos por canales rojo (R), verde (G), azul (B) y claridad (C). La lectura C se usa para normalizar y compensar variaciones de iluminación.

\subsection{Filtrado IQR (Interquartile Range)}
Se toma un conjunto de $n$ lecturas por canal (en este caso $n=20$) y se ordenan. Definimos:
\[
Q1 = \text{mediana del primer cuartil}, \quad Q3 = \text{mediana del tercer cuartil}
\]
\[
IQR = Q3 - Q1
\]
Los límites aceptables son:
\[
\text{lower} = Q1 - 1.5 \cdot IQR,\qquad \text{upper} = Q3 + 1.5 \cdot IQR
\]
Se promedian solo las muestras dentro de esos límites. Si no queda ninguna muestra, se utiliza la mediana como fallback.

\subsection{Centroides}
Para una clase con $n$ muestras y valores por canal $x_i$ (por ejemplo $r_i$), el centroide (media) se calcula como:
\[
\bar{r} = \frac{1}{n}\sum_{i=1}^{n} r_i
\]
Se calcula por cada canal (R, G, B, C) y se guarda en EEPROM como valores \texttt{uint16\_t}.

\subsection{KNN y distancia al cuadrado}
Para evitar cálculo de raíz cuadrada se emplea la distancia euclídea al cuadrado entre vectores de 3 componentes normalizados (r/c, g/c, b/c):
\[
d^2(\mathbf{x},\mathbf{y}) = (x_r - y_r)^2 + (x_g - y_g)^2 + (x_b - y_b)^2
\]
Se seleccionan los $K$ vecinos de menor distancia. Para desempates se utiliza el vecino global más cercano.

\subsection{Probabilidades a partir de pesos inversos}
Para proporcionar una probabilidad relativa por clase, se usan pesos inversos de distancia:
\[
w_i = \frac{1}{d_i + \varepsilon}
\]
Se suman pesos por clase y la probabilidad en \% es:
\[
P(\text{clase } j) = \frac{\sum_{i\in \text{clase } j} w_i}{\sum_{i} w_i}\times 100\%
\]
Si alguna distancia es cero (muestra idéntica al centroide), se maneja como caso especial dando peso directo a la(s) clase(s) con distancia cero.

\section{Explicación del código (resumen por bloques)}
A continuación se expone de forma compacta el propósito de los bloques principales del código final:

\subsection{Definiciones iniciales}
\begin{itemize}
  \item Se incluyen \texttt{Wire.h}, \texttt{Adafruit\_TCS34725.h}, \texttt{EEPROM.h}.
  \item Constantes importantes: \texttt{SAMPLES=20}, offsets de EEPROM (\texttt{ADDR\_RED}, \dots), \texttt{MAGIC\_BYTE}.
  \item Estructura \texttt{ColorVals} que agrupa \texttt{uint16\_t r,g,b,c}.
\end{itemize}

\subsection{Muestreo \& IQR}
Función \texttt{takeSampleIQR(\&r,\&g,\&b,\&c)}:
\begin{enumerate}
  \item Toma 20 lecturas: guarda arrays \texttt{rS, gS, bS, cS}.
  \item Ordena (insertion sort) y aplica rango IQR por canal.
  \item Devuelve valores limpios por referencia.
\end{enumerate}

\subsection{Acumulación y centroides}
\begin{itemize}
  \item En lugar de guardar todas las muestras en RAM, el código mantiene \texttt{sumR,sumG,sumB,sumC} y contadores \texttt{cnt} por clase.
  \item Al alcanzar mínimo (por ejemplo 3 muestras por clase) o al forzar con comando \texttt{W}, se calcula:
  \[
  cr = \frac{sumR}{cnt}, \ \text{etc.}
  \]
  y se escriben en EEPROM usando \texttt{writeU16}.
\end{itemize}

\subsection{Construcción del dataset KNN}
Tras escribir/leer centroides, el sketch normaliza por C (o por suma R+G+B si C=0) y guarda vectores normalizados en \texttt{datasetX} para uso en clasificación.

\subsection{Clasificación y probabilidades}
\begin{enumerate}
  \item En \texttt{classifyManual} se normaliza la muestra y se calculan las distancias (al cuadrado) frente a los centroides.
  \item Se eligen los $K$ más cercanos y se cuentan votos.
  \item Se calculan pesos inversos de distancia para estimar probabilidades por clase (manejo especial si alguna distancia es cero).
  \item Se imprimen los porcentajes por clase y se retorna la clase ganadora.
\end{enumerate}

\subsection{Medición de tiempo}
En el comando de clasificación (\texttt{C}) se mide tiempo total con \texttt{micros()}:
\[
\text{elapsed\_ms} = \frac{micros\_end - micros\_start}{1000.0}
\]
y se imprime el tiempo (ms) que tarda la lectura + clasificación.

\section{Cálculos y resultados (ejemplo)}
A modo de ejemplo muestro la plantilla de resultados que puedes completar con datos reales recogidos en experimentos:

\begin{table}[htbp]
\centering
\caption{Ejemplo de tabla de resultados (rellenar con datos experimentales).}


\begin{tabular}{llllllllll}
\cline{10-10}
ID & R  & G   & B  & C   & prediccion & P(roja) & P(Roja) & \multicolumn{1}{l|}{P(Verde)} & \multicolumn{1}{l|}{tiempo} \\ \cline{10-10} 
1  & 88 & 153 & 58 & 309 & verde      & 0\%     & 0\%     & 10\%                          & 62.124 ms                   \\
2  & 11 & 68  & 42 & 227 & naranja    & 16\%    & 80\%    & 4\%                           & 62.296 ms                   \\
3  & 98 & 38  & 30 & 165 & rojo       & 97\%    & 3\%     & 0\%                           & 61.872 ms                  
\end{tabular}
\end{table}

\subsection{Interpretación}
\begin{itemize}
  \item Las probabilidades permiten evaluar la confianza relativa del clasificador en cada decisión.
  \item El tiempo medido incluye la lectura hardware y la ejecución del KNN; con lecturas únicas se mantienen tiempos bajos (milisegundos).
\end{itemize}

\section{Conclusiones}
\begin{itemize}
  \item El uso de IQR en cada canal reduce de forma efectiva la influencia de outliers en lecturas del sensor.
  \item Calcular centroides vía sumas acumuladas reduce significativamente el uso de memoria RAM frente a almacenar todas las muestras.
  \item KNN con centroides (1 centroide por clase) favorece K=1; sin embargo, el uso de pesos inversos de distancia permite entregar una estimación probabilística útil.
  \item La estrategia implementada es adecuada para microcontroladores con recursos limitados, manteniendo precisión y capacidad de adaptación por recalibración.
\end{itemize}

\section*{Agradecimientos}
A los miembros del equipo y a las bibliotecas libres que hacen posible el proyecto (Adafruit, Arduino).

\vfill

\end{document}